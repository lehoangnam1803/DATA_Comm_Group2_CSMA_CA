% called by main.tex
\chapter{Kết luận}
\label{ch::chapter4}

Tổng kết lại, CSMA/CA là một kỹ thuật được sử dụng tại lớp MAC.
Kỹ thuật này rất quan trọng trong một hệ thống truyền thông không dây vì nó giảm thiểu được tỉ lệ bị mất hoặc hỏng gói tin vì quá trình xung đột.
So với CSMA/CD thì CSMA/CA có nhiều điểm nổi trội hơn. Thừa hưởng được những thế mạnh vốn có từ kỹ thuật tiền nhiệm, CSMA/CA cho ra hiệu suất cao hơn thời gian truyền dữ liệu ngắn hơn và
không đòi hỏi đến việc đảm bảo công suất của tín hiệu.

Từ kết quả mô phỏng có thể thấy, với kích thước gói tin mô phỏng lớn nhất là 80kb thì hiệu suất đạt trên 50\% với khoảng dưới 200 trạm.
Chiến lược 3 và 5 cho hiệu suất tốt nhất với tối thiểu 60\% cho gói tin có kích thước 80kb, hệ thống 400 trạm; hiệu suất đạt trên 95\% với gói tin 8kb và cùng số trạm là 400.
Kích thước gói tin tỉ lệ thuận với số gói tin được phân tách, thời gian truyền càng lâu
Hiệu suất kênh truyền còn phụ thuộc vào tốc độ của kênh truyền. Với bài mô phỏng này, tốc độ truyền được thiết lập ở mức 6Mbps và nếu tăng tốc độ này lên, hiệu suất cũng sẽ được tăng theo.
