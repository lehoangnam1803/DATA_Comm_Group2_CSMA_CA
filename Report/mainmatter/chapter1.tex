% called by main.tex
%
\chapter{Giới thiệu đề tài}
Trong hệ thống mạng truyền thông không dây, điển hình như Wi-Fi, đa truy cập người dùng là một vấn đề rất được quan tâm. Số lượng truy cập của người dùng tại mỗi thời điểm là khác nhau, có thể nói cách khác là ngẫu nhiên, nên xác suất các gói tin bị đụng độ hay xung đột khi đi đến các access point là tương đối cao. Trong mạng có dây, kỹ thuật CSMA/CD (Carrier Sense Multiple Access with Collision Detection) đã được sử dụng. Tuy nhiên kỹ thuật này đòi hỏi thời gian chờ giữa những lần gửi là tương đối cao và không phù hợp cho hệ thống mạng có nhiều người dùng cũng như là hệ thống mạng không dây.

Kỹ thuật CSMA/CA được phát triển dựa trên cơ chế cảm nhận môi trường truyền đã có của CSMA/CD và có sự cải tiến để phù hợp với mạng không dây. CSMA/CA đảm bảo giữa các thiết bị đầu cuối có sự thống nhất về thời gian chờ và gửi gói tin, không tạo ra sự xung đột trên đường truyền.

