% called by main.tex
%
\chapter{Giới thiệu đề tài}
Trong thế giới kỹ thuật mạng ngày nay, việc truy cập dữ liệu một cách hiệu quả và đồng bộ là một yếu tố cực kỳ 
quan trọng để đảm bảo hoạt động suôn sẻ của các hệ thống mạng. Đặc biệt, trong mạng không dây, nhu cầu điều phối 
truy cập dữ liệu trở nên càng phức tạp hơn do sự đa dạng và tính di động của các thiết bị. 

Trong kỹ thuật CDMA/CD, sự đụng độ được phát hiện bằng cách trạm phát sẽ nhận lại tín hiệu của chính nó và tín hiệu của
trạm phát khác. Điều này đòi hỏi tín hiệu phải duy trì được mức công suất cần thiết để đầu thu có thể phát hiện \cite{nam}. Tuy nhiên, 
trong môi trường truyền không dây, công suất của tín hiệu sẽ bị ảnh hưởng rất nhiều nên phương pháp này sẽ không đạt hiệu quả.
CSMA/CA (Carrier Sense Multiple Access with Collision Avoidance) đã được tạo ra với ý tưởng tránh xung đột trước khi truyền thay vì phát 
hiện xung đột sau khi truyền.

CSMA/CA và CSMA/CD đều là hai phương pháp truy cập phổ biến được sử dụng trong các mạng LAN (Local Area Network) để quản 
lý việc truyền dữ liệu và tránh xung đột. Tuy nhiên, chúng có các điểm khác biệt quan trọng. Trong CSMA/CD, các thiết bị 
phát hiện xung đột dữ liệu sau khi đã bắt đầu truyền. Nếu một xung đột được phát hiện, các thiết bị sẽ ngừng truyền và gửi 
một tín hiệu để thông báo cho các thiết bị khác biết về xung đột, sau đó chờ một khoảng thời gian ngẫu nhiên trước khi thử lại.
Trong CSMA/CA, các thiết bị tránh xung đột dữ liệu bằng cách gửi các gói tin yêu cầu trước khi truyền dữ liệu thực sự. Nếu không 
nhận được phản hồi xác nhận từ thiết bị đích, nó sẽ giả định rằng có xung đột và sẽ chờ một khoảng thời gian ngẫu nhiên trước khi thử lại.

Một điểm đặc biệt của CSMA/CA là việc tránh xung đột dữ liệu thực hiện thông qua quá trình tránh va chạm. Thiết bị gửi dữ liệu sẽ 
gửi một gói tin RTS (Request to Send) cho thiết bị đích, sau đó thiết bị đích sẽ gửi lại một gói tin CTS (Clear to Send) xác nhận. 
Trong khoảng thời gian này, các thiết bị khác sẽ tạm ngưng truyền dữ liệu để tránh xung đột. Quá trình này giúp giảm thiểu xác suất 
xung đột dữ liệu và tăng hiệu suất của mạng không dây.

Tóm lại, CSMA/CA là một phương pháp truy cập thông minh được sử dụng rộng rãi trong các mạng không dây, giúp điều phối việc truyền dữ 
liệu một cách hiệu quả và đồng bộ, đồng thời giảm thiểu xác suất xung đột dữ liệu và tăng cường hiệu suất mạng.

Trong bài báo cáo này, nhóm chúng em thực hiện mô phỏng kỹ thuật CSMA/CA cho hệ thống có từ 10 đến 400 nodes; gói tin gửi có kích thước từ 8000
đến 80000 bits.